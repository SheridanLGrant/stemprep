\documentclass[12pt]{article}

\usepackage[margin=1in]{geometry}
\usepackage{amsmath,amsthm,amssymb,amsfonts}

\newcommand{\Pp}{\emph{\emph{P}}}

\begin{document}

\title{Statistics Homework 3}
\author{Mr. Grant}
\date{}
\maketitle

\textbf{Due Date:} Thursday, July 13th \\

Turn in this homework by emailing me (SLGstats@uw.edu) an R file with these four functions in it. The file should be named [firstname]\_[lastname]\_hw3.R. \\

We didn't get to this in class today, but functions can either ``print'' or ``return.'' You write the function almost the exact same way. So an \textbf{R} function that ``takes in'' something and \textbf{prints} something plus two would be:
\begin{verbatim}
add2 <- function(something) {
	print(something + 2)
}
\end{verbatim}
while an \textbf{R} function that takes in something and \textbf{returns} something plus two would be
\begin{verbatim}
add2 <- function(something) {
	return(something + 2)
}
\end{verbatim}

Some of these questions require you to use R to simply compute a number. You may ask your friends for help or hints on how to get the answer, but \textbf{do not} just copy their answers.

\begin{enumerate}
	\item{The ``natural numbers'' or the ``positive integers'' refer to the whole numbers that are greater than 0. So the first natural number is 1. What is the sum of the square roots of the first 10,000 natural numbers?}
	\item{Among the tangents of the integers from 100 to 200, what is the smallest?}
	\item{``Mean'' is another way of saying ``average.'' What is the mean of all the odd numbers between 1000 and 2000?}
	\item{Write an R function for the mathematical function $f(x) = 4x^2 + 3x + \sqrt{x} + \sin(x) + 7$.}
	\item{Write a function that takes in a vector and returns the first 5 elements of that vector, or the phrase ``Vector too short!'' if the vector has length less than 5.}
	\item{\textbf{CHALLENGE:} Write a function that takes in a whole number $n$ and returns the sum of all the positive odd numbers less than or equal to $n$.}
	\item{\textbf{CHALLENGE:} Write a function that takes in three numbers and returns the ``middle'' number. That is, the function should take in inputs $a$, $b$, and $c$, and return $a$ if $b \le a \le c$ or $c \le a \le b$, etc.}
\end{enumerate}
\end{document}
