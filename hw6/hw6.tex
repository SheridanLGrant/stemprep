\documentclass[12pt]{article}

\usepackage[margin=1in]{geometry}
\usepackage{amsmath,amsthm,amssymb,amsfonts,graphicx}
\usepackage{hyperref}

\newcommand{\Pp}{\emph{\emph{P}}}

\begin{document}

\title{Statistics Homework 6}
\author{Mr. Grant}
\maketitle

\textbf{Due Date:} Wednesday, July 26th \\

This homework uses the data sent out after Thursday's class, \textit{2013\_incarceration\_fixed.csv} and \textit{2013\_population\_race\_fixed.csv}. You must turn in, via email, a typed report of between 250 and 500 words, \textbf{not including graphs and tables}. The report must include at least 1 graph and at least 1 table. You must also turn in (via email) an R file containing the code you used to answer the questions and generate graphs. \textbf{Attach the report and the R file to the same email}--that way I'm not sifting through, like, hundreds of emails to figure out who turned in what.

You are to address the following two questions:
\begin{enumerate}
	\item Are people of different races incarcerated at different rates in state prisons in the United States?
	\item If so, can you identify ages at which the disparities in incarceration rates between different races are most extreme?
\end{enumerate}
Use the programming techniques we have learned to help you find the answers to these questions. Also note that the ``rate'' at which people of a certain race go to prison is the same as the probability of a person of that race going to prison. For example, if 10\% of white people go to prison (10\% is the rate at which white people go to prison), then the probability of a typical white person going to prison is 10\%.

\end{document}
